\textbf{Write heavy workloads with LSM.} The rising research in the field of robotics and IoT devices, combined with the 
integration of artificial intelligence and machine learning, has resulted in the generation of vast amounts of data.
This data often requires real-time processing and exhibits a write-heavy nature. Data stores like RocksDB and LevelDB
have been designed to efficiently handle such workloads. These data stores rely on the technique of
\textbf{log-structured merge (LSM)} trees, an efficient data structure tailored for managing write-heavy workloads. The
fundamental concept underlying LSM trees is of out-of-place updates, which logically invalidate keys 
instead of performing in-place updates.\\
\textbf{Compaction.} The LSM trees are composed of multiple levels, each of which is a sorted run of key-value pairs. 
The compactions are performed to merge the sorted runs from the lower levels into the higher levels. The process is 
triggered when the size of a level exceeds a certain threshold. It helps in removing the stale data from LSM and 
making room for new data in lower levels.\\
\textbf{Range queries.} The LSM trees are optimized for write-heavy workloads, but it also supports range queries. The
range queries are performed by merging the sorted runs from multiple levels and filtering out the keys that have been
logically invalided. This process is also called \textbf{sort-merge}, which is similar to the compaction.\\
\textbf{Problem.}