\textbf{Write heavy workloads with LSM.} The rising research in the field of robotics and IoT devices, combined with the 
integration of artificial intelligence and machine learning, has resulted in the generation of vast amounts of data.
This data often requires real-time processing and exhibits a write-heavy nature. The data stores like RocksDB and 
LevelDB have been designed to efficiently handle such workloads. These data stores rely on the technique of
\textbf{log-structured merge (LSM)} trees, an efficient data structure tailored for managing write-heavy workloads. The
fundamental concept underlying LSM trees is of out-of-place updates, which logically invalidate keys instead of 
performing in-place updates.

\textbf{Compaction.} The LSM trees are composed of multiple levels, each of which is a sorted run of key-value pairs. 
The compactions are performed to merge the sorted runs from the lower levels into the higher levels. The process is 
triggered when the size of a level exceeds a certain threshold. It helps in removing the stale data from LSM and 
making room for new data in lower levels.

\textbf{Range queries.} The LSM design is optimized for write-heavy workloads, but it also supports range queries. The
range queries are performed by merging the sorted runs from multiple levels and filtering out the keys that have been
logically invalided. This process is also called \textbf{sort-merge}, which is similar to the compaction~\cite{Sarkar2021}.

\textbf{Problem.} When executing a range query, the sort-merge operation is performed on sorted runs from multiple 
levels, leading to the retrieval of both valid and invalid keys from higher levels. Consequently, more data is read than 
actually required. This situation is acceptable when performing a single range query for a specific range. However, when 
the same query or an overlapping one is executed repeatedly, a nearly identical amount of work is executed. Furthermore, 
when a compaction is triggered within that specific range, some of the invalid keys that were previously read, sorted 
and filtered during the range query are revisited. This results in redundant CPU cycles and I/O operations, where the 
same data bytes are read repetitively until reaching the last level.


% Motivation
\subsection{Motivation}
The repetition of work involving invalid keys can be effectively mitigated by redirecting valid keys, filtered through 
the sort-merge operation during a range query, back to the higher levels. This approach can be termed as 
\textbf{query-driven compaction}. By minimizing the presence of invalid keys within the LSM tree, the compaction process 
gains efficiency, subsequently leading to less number of I/O operations and a more optimal utilization of CPU cycles.


% Problem Statement
\subsection{Problem Statement}
The state-of-the-art LSM-based data stores performs range queries by retrieving multiple files from various levels and 
filter out invalid keys. Once the range query is complete, the work done via the sort-merge operation for query is 
discarded. The results are neither cached nor flushed back to the LSM tree except returning it to the application. This 
approach can give rise to two problems: (1) \textit{Redundant Work}: when the same range query or 
an overlapping one is executed again, and (2) \textit{Increased Write Amplification}: when a compaction is triggered 
for the files containing keys within the range of a previous range query. During these process, the system re-reads the 
invalid keys and subsequently either drops them or replaces them with new values. As a result, the same data bytes are 
read and written during both the range query and compaction processes, leading to higher read, write, and space 
amplification.


% Contributions
\subsection{Contributions}
In this paper, we present a query-driven compaction strategy that involves writing the valid keys back to the higher 
levels of the LSM tree. While this approach may slightly increase the flush write bytes during a range query, it 
substantially reduces the presence of invalid keys in the LSM tree. As a result, there is a notable reduction in read, 
write, and space amplification during both compaction processes and future range queries. We have also implemented our
approach in RocksDB, a popular LSM-based data store and conducted a series of experiments to evaluate its performance.