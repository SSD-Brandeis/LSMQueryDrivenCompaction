In this study, we explored and addressed key challenges in log-structured merge (LSM) trees, particularly focusing on 
improving the efficiency of range queries and compactions in the context of write-intensive workloads. The prevalent use 
of LSM trees in modern data systems prompted our investigation into optimizing the traditional approach through the 
introduction of a novel query-driven compaction strategy. The motivation behind our work stemmed from the observation
that conventional LSM designs incur redundant work and increased write amplification, especially when executing 
overlapping range queries. To tackle these issues, we proposed a query-driven compaction strategy that selectively 
removes invalid keys during range queries, significantly reducing the workload for subsequent queries. Our approach 
demonstrated substantial benefits in experimental settings, showcasing a notable reduction in the number of compactions 
triggered during workload execution. We empirically validated the efficiency of our strategy through experiments 
conducted on a real-world LSM-based data store, RocksDB, using baremetal cloud instances. The results highlighted 
significant improvements in both speed and efficiency, particularly evident in reduced space amplification and optimized 
use of system resources.

We acknowledged the limitations of a blind application of the query-driven approach and introduced an informative 
compaction strategy to make more informed decisions based on the overlap between levels. This adaptive approach 
showcased comparable write amplification to the conventional method while providing additional advantages such as 
reduced space amplification. The experimental findings demonstrated the effectiveness of our approach, indicating that 
the query-driven compaction outperformed traditional methods in terms of space amplification. Through a 
carefully designed experimental setup involving different workload scenarios and tuning parameters, we showcased the 
versatility and applicability of our strategy.

\subsection{Concrete Benefits}

\Paragraph{Space Efficiency}
RQDC consistently reduces the overall size of the database and the number of SST files. This directly translates to 
more efficient space utilization.

\Paragraph{Query Performance}
The reduction in invalid keys during compaction positively impacts subsequent range query executions, leading to 
improved query performance.

\Paragraph{Competitive Write Amplification}
RQDC manages to achieve competitive write amplification level compared to the vanilla approach, despite introducing 
additional compaction writes.

\subsection{Future Directions}

While the results are promising, there are opportunities for further refinement and optimization:

\Paragraph{Fine-Tuning Parameters}
Experimentation with different values of \textit{lower\_bound} and \textit{upper\_bound} should continue to find the 
most effective thresholds for specific size ratios and workloads. Dynamic adjustment of these parameters based on 
workload characteristics could enhance adaptability.

\Paragraph{Finding More Accurate Overlap}
Discovering a more accurate representation of overlap before initiating compaction could eliminate cases where the 
overlap is merely a result of start and end keys. Identifying the actual keys, potentially alternate keys on both 
adjacent levels (e.g., odd and even), can contribute to reducing such scenarios and saving on write amplification.

\Paragraph{Experimenting Deletes}
Extending the RQDC approach to handle delete operations, especially in the presence of tombstones, has the potential 
to further reduce write amplification.

In conclusion, our research contributes to the ongoing efforts to enhance LSM-based data stores for write-intensive 
workloads. The query-driven compaction strategy, coupled with an informative decision-making approach, tries to offers a 
practical solution to the challenges posed by overlapping range queries. Our findings provide valuable insights for 
database developers and researchers seeking to optimize the performance of LSM trees in real-world scenarios.
