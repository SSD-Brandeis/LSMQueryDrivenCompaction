In modern data systems, particularly for write-intensive workloads, log-structured merge (LSM) trees have become the most 
widely used technique. The idea of out-of-place updates, which logically 
invalidate keys that may exists on multiple levels, helps LSM-trees excel for write heavy workloads. However, current 
LSM-tree designs do not capitalize on the sort-merge operations performed for the range queries, which leads to carrying 
out almost the same amount of work for each range query, even when they are same or overlapping. This inefficiency can 
result in the wastage of CPU cycles and unnecessary I/O operations in worst-case scenarios due to presence of logically 
invalid keys. \cite{sarkar22}

In this paper, we present a query-driven compaction strategy that removes the invalid (logically deleted) keys from the 
LSM-tree and flushes back the valid keys filtered by sort-merge performed during a range query. We conduct performance 
experiments with the help of a custom implementation in one of the popular LSM-based data stores, RocksDB.
